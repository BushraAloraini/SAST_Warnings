\section{Data extraction} {{{2
to extract data necessary for the analysis 
of source code vulnerability evolution. 
The data extraction process consists of the sequence of five steps:

\begin{itemize}
\item  \textbf{Step 1: snapshots extraction}
As we are working on GitHub repositories, 
so we will be identifying two snapshots: 
one in 2012 and the other in 2017. 

\item \textbf{Step 2: identification of vulnerable source code lines}  
We will be running all SATs on 2012 and 2017 versions, 
however we are interested in all vulnerabilities that 
were flagged in 2012 and removed from the code in 2017. 
This analysis will help us to identify the set of source code lines
that contain warnings on both 2012 and 2017 versions. 
The output of this step is, for each snapshot, 
the list of vulnerable source code lines with
a vulnerability description as extracted by the tool, 
and a vulnerability classification according to the used vulnerabilities taxonomy. 

\item \textbf{Step 3: differences identification and line tracing} 
To analyze the evolution of vulnerable source code lines over snapshots, 
we need to identify in each tested file the addition of new lines, 
in the removal of existing lines, and the change of existing lines. 
Therefore, we will be using tracing tools such as ldiff tool. 

\item \textbf{Step 4: determining vulnerability changes among snapshots}
In this research question, we could identify false positive and true positive warnings, 
and this by considering the following cases:
	Any vulnerability warning that was reported in 2012 and 2017 
    will be consider to be false positives.
	Any vulnerability warning that was reported in 2012 but not 2017 
    will be considered for further analysis: 
    1- when  a source code line containing a vulnerability was removed, 
    2- the source code line has changed, 
    or 3-  the source code line  has not been changed but a change occurred somewhere else.

\item \textbf{Step 5: analyzing documentation of vulnerability removal/ disappear} 
To better understand in what context a vulnerability was removed or disappeared,
we will be using automatic analysis on all the commit messages
that indicate vulnerability fix following \cite{mockus2000identifying} method. 
\end{itemize}
