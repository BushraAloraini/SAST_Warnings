\begin{itemize}
\item \textbf{RQ1: How effective the studied static analysis tools are?}
We  measure the effective of he studied static analysis tools in terms of precision, false positive rate, running time. We found that RATS outperform other tools regarding the running time. While Flawfinder found to be more precise and has less false positive rate, though Cppcheck did not produced any buffer overflow warning.
\item \textbf{RQ2: How true positive vulnerabilities evolve over time?}
We focus here on vulnerabilities age. We found that  most of the true positive vulnerabilities were detected and deleted in less than 6 months period of time. However, there were some vulnerabilities that up to aged 71 months, about 5 years and 9 months, and then finally removed. The result need further investigation using more advanced static analysis tools to confirm this finding.

\item \textbf{RQ3: What are the patterns of buffer overflow bugs in which they were detected by static analysis tools?}
We focus here on vulnerabilities frequency. We found that calling potentially dangerous functions to be the most frequent pattern that occurred in true positive warnings. Other categories that happed frequently as well are fixed buffer size and unsensitized input from untrusted sources
\end{itemize}


We believe that conducting this study will gave us a better understanding of the pattern of emerging buffer overflow vulnerabilities. In addition, at the end of this study we will able to recognize which static analysis methods perform better. Thus, we will be able to build a better a tool to discover such vulnerability.
